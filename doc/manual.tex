\documentclass[a4paper,11pt]{article}
\usepackage[utf8x]{inputenc}
\usepackage[OT1]{fontenc}
\usepackage[english]{babel}

\usepackage{amsmath}
\usepackage[charter]{mathdesign}

\author{Simon Mitternacht}
\date{\today}
\title{Sasalib Manual}

\begin{document}
\maketitle

The program/library Sasalib, including this manual, is licensed under
GPLv3, and can be downloaded from the github at: 
\begin{center}
\texttt{https://github.com/mittinatten/sasalib/}
\end{center}

This is still a draft and not complete in any way.

\newpage
\section{Introduction}

Sasalib is a program to calculate solvent accessible surface areas
(SASA) for proteins. It can both be used as a standalone command line
program or as a C library. The library interface allows the user to
set most parameters of the calculation, and returns the SASA of the
individual atoms, allowing summation of SASA over user-defined
atom-classes. The library implements both Lee \& Richards' and Shrake
\& Rupley's algorithms.

To author's knowledge there aren't any fully open source programs for
calculating solvent accessible surface areas (SASA) of
proteins. Neither are there any packages designed to be easily
integrated as a library in other programs. 

The SASA is defined as the surface area of protein accessible to a
spherical probe rolling over the surface of the protein, including
internal cavities. The probe represents a solvent molecule, and
therefore cavities that are too small for the solvent to enter are not
counted at all. The position of the center of the probe sphere is used
to define the surface area, not the point of contact.

\section{Algorithms}

There are two classical approximate algorithms that can be used to
calculate SASA, by Lee \& Richards and Shrake \& Rupley
respectively. Below are brief presentations of both. 

An atom $i$ has a van der Waals radius $r_i$, the surface probe has
radius $r_\text{P}$ and when these are added we get $R_i = r_i +
r_\text{P}$. The SASA is calculated by adding up the surface patches
of the spheres of radius $R_i$ that are not overlapping with any other
sphere.

\subsection{Lee \& Richards}

Divide the protein into slices of thickness $\delta$ along an
axis. The position of the slice along that axis is denoted $z$. The
position of the center of atom $i$ along the same axis is $z_i$. In
the slice, each atom is thus a circle of radius $R_i^\prime =
\sqrt{R_i^2-(z-z_i)^2}$. Some of these circles are completely buried,
completely exposed, or partially exposed.

The exposed arc lengths for each atom can be calculated exactly. For
each pair of atoms $i,j$, the distance between their centers is
$d_{ij}$. If $d_{ij} < R_i^\prime + R_j^\prime$, there is an
overlap. If $d_{ij} < R_j^\prime - R_i^\prime$ circle $i$ is
completely inside $j$, and the other way around. If $d_{ij}$ lies
between these two cases the angle of circle $i$ that is buried due to
circle $j$ is $$\alpha = 2\arccos \bigl[({R_i^\prime}^2 + d_{ij}^2 -
  {R_{j}^\prime}^2)/(2R_i^\prime d_{ij})\bigr].$$ The middle point of
the arc on the circle can be denoted as an angle $\beta$, and thus the
arc spans the interval $[\beta-\alpha/2,\beta+\alpha/2]$. By adding up
these arcs and taking into account any overlap between them we get
the total buried angle $\gamma_i$ of circle $i$. The exposed arc length in
this slice is thus $L_i = R_i^\prime(2\pi-\gamma_i)$.

The contribution to the SASA from each slice is $$ S_\delta =
\sum_{i \in \text{slice}}L_i\Delta_i $$ where
$$
  \Delta_i = \frac{R_i}{R_i^\prime} \Biggl[\frac{\delta}{2} 
    + \min\Biggl(\frac{\delta}{2},R_i -
    \lvert z - z_i \rvert\Biggr)\Biggr]. 
$$ 
Finally, the total SASA is obtained by adding up the contribution from
all the slices, either for the whole protein, or atom by atom.

\subsection{Shrake \& Rupley}

For each atom $i$, use a set of test points evenly distributed
(approximately) over the sphere of radius $R_i$, and count how many of
the test points are not inside any of the other extended spheres of
radius $R_j$. The number of exposed testpoints divided by the total
number of test points gives the exposed solid angle of that atom.

\section{Implementation}

\section{Using Sasalib}

\subsection{Installing}

The repository can be cloned from the github either using git
directly with the command
\begin{verbatim}
    $ git clone https://github.com/mittinatten/sasalib.git
\end{verbatim}
or by downloading the zipped archive from
\begin{verbatim}
    https://github.com/mittinatten/sasalib/archive/master.zip
\end{verbatim}
Since Sasalib only depends on regular C and GNU libraries most users
will be able to compile it by simply typing \texttt{make}\footnote{Has
  been tested on Linux and Mac OS X machines.}. If any other compiler
than the Gnu C Compiler is preferred the makefile will need to be
changed accordingly.

\subsection{Stand-alone program}

Compilation creates the binary \texttt{calc\_sasa}, which can be used
to calculate the SASA of a PDB-file. By default the program reads the
PDB from STDIN
\begin{verbatim} 
    $ calc_sasa < PDB-file    
\end{verbatim}
The PDB-file can also be specified using the flag \texttt{-f}
\begin{verbatim}
    $ calc_sasa -f PDB-file
\end{verbatim}
Other options for running are displayed with the flag
\texttt{-h}. These options allow the user to specify which algorithm
to use, and parameters for the algorithm. By default the Shrake \&
Rupley algorithm is used, with 100 test points. This option gives a
fast calculation, for high precision the number of test points should
be increased.
 
\subsection{Library interface}

To calculate the SASA of a protein using the library interface the
following steps need to be followed:
\begin{enumerate}
  \item Initiate a protein. Either by reading a pdb-file or by adding
    atoms manually one-by-one.
  \item Give each atom a radius. Sasalib can calculate atomic radii
    according to Ooi et al (OONS-radii) for the 20 canonical amino
    acid types. The user can also specify the radius for each
    individual atom. The two methods can be combined if a protein has
    a few non-standard atoms.
  \item Perform the SASA calculation, using the algorithm of choice.
  \item Integrate the results of the SASA calculation to get for
    example the polar and apolar surface areas. This can be done using
    some default setups based on the OONS classification, or by the
    user.
\end{enumerate}

\section{Known issues}

The atoms of non-standard amino acids will be labelled unknown type,
and their contribution to for example the polar/apolar area will have
to be integrated manually by the user.

\section{Ideas for improvement and extension}

Perl and Python bindings.

Other, faster or more exact algorithms.

\end{document}
