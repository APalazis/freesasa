\documentclass[a4paper,11pt]{article}
\usepackage[utf8x]{inputenc}
\usepackage[OT1]{fontenc}
\usepackage[english]{babel}
\usepackage[margin=1.5in]{geometry}
\usepackage{amsmath}
\usepackage{cite}

%\usepackage{fancyvrb}

\author{Simon Mitternacht}
\date{\today}
\title{Sasalib Manual}

\begin{document}
\maketitle
\noindent
The program/library Sasalib, including this manual, is licensed under
GPLv3, and can be downloaded from the github at: 
\begin{center}
\texttt{https://github.com/mittinatten/sasalib/}
\end{center}

This document is still a draft and not complete in any way.

\newpage
\section{Introduction}

Sasalib is a program for calculating solvent accessible surface areas
(SASA) for proteins. It can both be used as a standalone command line
program or as a C library. The library interface allows the user to
set most parameters of the calculation, and returns the SASA of the
individual atoms, allowing summation of SASA over user-defined
atom-classes. The library provides both the Lee \& Richards
\cite{LnR} and Shrake \& Rupley \cite{SnR} algorithms.

To the author's knowledge there aren't any fully open source programs for
calculating solvent accessible surface areas (SASA) of
proteins. Neither are there any packages designed to be easily
integrated as a library in other programs. 

The SASA is defined as the surface area of a protein accessible to a
spherical probe rolling over the protein, including internal
cavities. The probe represents a solvent molecule (i.e. water), and
therefore cavities that are too small for the solvent to enter are not
counted at all. The position of the center of the probe sphere is used
to define the surface area, not the point of contact.

\section{Algorithms}

There are two classical approximate algorithms that can be used to
calculate SASA, by Lee \& Richards \cite{LnR} and Shrake \& Rupley
respectively \cite{SnR}. Both can be calculated to arbitrary precision
by refining the resolution.

We will use the following notation: An atom $i$ has a van der Waals
radius $r_i$, the surface probe has radius $r_\text{P}$ and when these
are added we get $R_i = r_i + r_\text{P}$. The SASA is calculated by
adding up the surface patches of the spheres of radius $R_i$ that are
not overlapping with any other sphere.

\subsection{Lee \& Richards}

Lee \& Richards' algorithm calculates the surface area by slicing the
protein, calculating the length of the solvent exposed contours in
each slice and then adding up the length multiplied by the slice
thickness. Precision is increased by making the slices thinner and the
calculation time scales as the number of slices.

Divide the protein into slices of thickness $\delta$ along an
arbitrary axis. The position of the slice along that axis is denoted
$z$. The position of the center of atom $i$ along the same axis is
$z_i$. In the slice, each atom is thus a circle of radius $$R_i^\prime
= \sqrt{R_i^2-(z-z_i)^2}\,.$$ These circles are either completely buried,
completely exposed, or partially exposed.

The exposed arc lengths for each atom can be calculated exactly. For
each pair of atoms $i,j$, the distance between their centers is
$d_{ij}$. If $d_{ij} > R_i^\prime + R_j^\prime$, there is no
overlap. If $d_{ij} < R_j^\prime - R_i^\prime$ circle $i$ is
completely inside $j$, and the other way around. If $d_{ij}$ lies
between these two cases the angle of circle $i$ that is buried due to
circle $j$ is $$\alpha = 2\arccos \bigl[({R_i^\prime}^2 + d_{ij}^2 -
  {R_{j}^\prime}^2)/(2R_i^\prime d_{ij})\bigr].$$ The middle point of
the arc on the circle is at an angle $\beta$ in circle $i$, and thus the
arc spans the interval $[\beta-\alpha/2,\beta+\alpha/2]$. By adding up
these arcs and taking into account any overlap between them we get
the total buried angle $\gamma_i$ of circle $i$. The exposed arc length in
this slice is thus $L_i = R_i^\prime(2\pi-\gamma_i)$.

The contribution to the SASA from each slice is $$ S_\delta =
\sum_{i \in \text{slice}}L_i\Delta_i $$ where
$$
  \Delta_i = \frac{R_i}{R_i^\prime} \biggl[\frac{\delta}{2} 
    + \min\biggl(\frac{\delta}{2},R_i -
    \lvert z - z_i \rvert\biggr)\biggr]. 
$$ 
Finally, the total SASA is obtained by adding up the contribution from
all the slices, either for the whole protein, or atom by atom.

\subsection{Shrake \& Rupley}

For each atom $i$, use a set of test points evenly distributed
(approximately) over the sphere of radius $R_i$, and count how many of
the test points are not inside any of the other extended spheres of
radius $R_j$. The number of exposed testpoints divided by the total
number of test points gives the exposed solid angle of that atom. The
precision of the algorithm is increased by increasing the number of
test points, and calculation time scales linearly with the number of
test points.

\subsection{Comparison of the two}

Speed as function of $N$ for a given accuracy and/or accuracy as a
function of speed.

\section{Implementation}

\section{Using Sasalib}

\subsection{Installing}

The repository can be cloned from the github either using git
directly with the command
\begin{verbatim}
    $ git clone https://github.com/mittinatten/sasalib.git
\end{verbatim}
or by downloading the zipped archive from
\begin{verbatim}
    https://github.com/mittinatten/sasalib/archive/master.zip
\end{verbatim}
Since Sasalib only depends on regular C and GNU libraries most users
will be able to compile it by simply typing \texttt{make}\footnote{Has
  been tested on Linux and Mac OS X machines.}. If any other compiler
than the Gnu C Compiler is preferred the makefile will need to be
changed accordingly.

\subsection{Stand-alone program}

Compilation creates the binary \texttt{calc\_sasa}, which can be used
to calculate the SASA of a PDB-file. By default the program reads the
PDB from STDIN
\begin{verbatim} 
    $ calc_sasa < PDB-file    
\end{verbatim}
The PDB-file can also be specified using the flag \texttt{-f}
\begin{verbatim}
    $ calc_sasa -f PDB-file
\end{verbatim}
Other options for running are displayed with the flag
\texttt{-h}. These options allow the user to specify which algorithm
to use, and parameters for the algorithm. By default the Shrake \&
Rupley algorithm is used, with 100 test points. This option gives a
fast calculation, for high precision the number of test points should
be increased. 

\subsection{Library interface}

Sasalib provides functions for doing the following steps in a SASA
calculation:
\begin{enumerate}
  \item Initialize protein.
  \item Calculate atomic radii.
  \item Calculate SASA.
  \item Output results grouping atoms/residues in different ways.
\end{enumerate}
The third step is the core functionality of the library and can be
used separately if users wish to handle the other steps
themselves. Below follows a brief description for the functionality
provided for each of the four steps, but first a sample program
showing a complete workflow.

\subsubsection{Sample program}
The following lines read a PDB-structure from a file, calculates
SASA using Lee \& Richards' algorithm and outputs the values for the
polar and apolar atoms respectively.
\begin{verbatim}
#include <stdio.h>
#include <stdlib.h>
#include "protein.h"
#include "sasa.h"

int main(int argc, char **argv) {
    // initialization from STDIN
    protein *p = protein_init_from_pdb(stdin);

    // allocate memory
    double *sasa = (double*) malloc(sizeof(double)*protein_n(p));
    double *r = (double*) malloc(sizeof(double)*protein_n(p));

    // calculate atomic radii
    protein_r_def(p,r);
    
    // SASA calculation using Lee & Richards with 
    // slices of width 0.25 Å
    sasa_lee_richards(sasa,protein_xyz(p),r,protein_n(p),0.25);

    // Output SASA for polar/apolar atoms
    sasa_per_atomclass(stdout,oons_classes(),p,sasa);
   
    // free memory
    protein_free(p);
    free(sasa);
    free(r);

    return 0;
}
\end{verbatim}

\subsubsection{Initiate a protein}

A protein is represented by the struct \texttt{protein}, which is
declared in the header \texttt{protein.h}. A proteins can be
initialized either by reading a PDB-file or by adding atoms manually
one-by-one.

The following function initializes a protein from a PDB-file
\begin{verbatim}
    protein* protein_init_from_pdb(FILE*); 
\end{verbatim}
The pointer is freed using
\begin{verbatim}
    void protein_free(protein*);
\end{verbatim}
An empty protein struct is allocated with 
\begin{verbatim}
    protein* protein_init();
\end{verbatim}
which can then be used to add atoms one by one
\begin{verbatim}
    void protein_add_atom(protein *p, 
                          const char* atom_name,
                          const char* residue_name, 
                          const char* residue_number,
                          char chain_label,
                          double x, double y, double z);

\end{verbatim}
for example like this
\begin{verbatim}
    protein *p = protein_init();
    protein_add_atom(p, " CA ", "GLY", "   1", "A",
                     x, y, z);
\end{verbatim}
Here the labels for the atom are as they would be in a PDB file,
including whitespace. The atom name is a 4-character string, like the
second argument above, the residue name a 3-character string, like the
third argument, the residue number has 4 characters, and the chain
label 1. The last three arguments are the coordinates of the atom. The
reason residue numbers are not represented as integers here, is that
in some PDB files have number sequences such as 1A, 1B, 1C, 1D, 2, 3,
4, \ldots.

The rationale for directly using the strings found in PDB files is
twofold. (i) Most programs that handle proteins will have routines for
generating PDB-files and it should thus be fairly straightforward to
integrate this library into such a program. (ii) Non-standard atoms
can be added with the labels they have in the input, making
sub-sequent analysis more straightforward. This means that no analysis
is performed of the atom types, to check for consistency, etc. The
atom types are only used later for assigning radii to the atoms,
something that can also be done manually.

\subsubsection{Assign atomic radii}
Give each atom a radius. Sasalib can calculate atomic radii according
to Ooi et al (OONS-radii) for the 20 canonical amino acid types. The
user can also specify the radius for each individual atom. The two
methods can be combined if a protein has a few non-standard atoms.

Functions for OONS-classification are found in \texttt{oons.h}. There
are two versions
\begin{verbatim}
    double oons_radius_pdbline(const char *pdb_line);
    double oons_radius(const char *res_name,
                       const char *atom_name);
\end{verbatim}
where the argument in the first version is a line from a PDB-file, and
in the second case residue name and atom name according to the above
(e.g. \texttt{"GLY", " CA "}). To get all radii for a protein in one
go, use
\begin{verbatim}
    void protein_r_def(const protein *p, double *r);
\end{verbatim} 
Here the array \texttt{r} is assumed to be of the same size as the
protein, i.e. the user is responsible for allocating and freeing the
memory. The function
\begin{verbatim}
    int protein_n(protein *p);
\end{verbatim}
A user-defined atom-classification scheme can be used by calling 
\begin{verbatim}
    void protein_r(const protein *p, 
                   double *r,
                   double (*atom2radius)(const char *res_name, 
                                         const char *atom_name));
\end{verbatim}
The function \texttt{protein\_r\_def}, defined above, uses
\texttt{protein\_r} internally. It consists of the single line
\begin{verbatim}
    protein_r(p,r,oons_radius);
\end{verbatim}
which illustrates the use of the functions \texttt{protein\_r} and
\texttt{oons\_radius}.

\subsubsection{Perform SASA calculations}
Perform the SASA calculation, using the algorithm of choice, calling
the functions in \texttt{sasa.h}
\begin{verbatim}
    void sasa_shrake_rupley(double *sasa,
                            const vector3 *xyz,
                            const double *radii,
                            size_t n_atoms,
                            int n_points);
    void sasa_lee_richards(double* sasa,
                           const vector3 *xyz,
                           const double *radii,
                           size_t n_atoms,
                           double grid);

\end{verbatim}
The protein struct is deliberately

\subsubsection{Output results}
Integrate the results of the SASA calculation to get for example the
polar and apolar surface areas. This can be done using some default
setups based on the OONS classification, or by the user.

\section{Known issues}

The atoms of non-standard amino acids will be labelled unknown type,
and their contribution to for example the polar/apolar area will have
to be integrated manually by the user.

\section{Ideas for improvement and extension}

In no specific order
\begin{itemize}
\item Perl and Python bindings.
\item Other, faster or more exact algorithms.
\item Molecular surface calculations?
\item Add a library of commonly seen atoms, such as those in capping
  end groups and in phosphorylated amino acids.
\item Replacing the B-values in a PDB-file with the SASA values for
  each atom.
\item S \& R, and parts of L \& R, can be trivially
  parallelized. There is probably no sense in doing a large scale
  parallelization, but making the calculations multithreaded would
  allow the users to take advantage of multicore processors.
\item Interface to download PDB-file and calculate SASA given only PDB
  code.
\end{itemize}

\begin{thebibliography}{50}

\bibitem{LnR} 
  Lee B, Richards FM (1971) The interpretation of protein structures: estimation of static accessibility. Journal of molecular biology 55: 379–400.

\bibitem{SnR} 
  Shrake A, Rupley JA (1973) Environment and exposure to solvent of protein atoms. Lysozyme and insulin. Journal of Molecular Biology 79: 351–371.
\end{thebibliography}

\end{document}
